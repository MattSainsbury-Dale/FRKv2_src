\documentclass[english]{letter}
\usepackage[latin1]{inputenc}
\usepackage{amsmath}
%\usepackage{caption}
%\usepackage{subfig}
\usepackage{graphicx}
\usepackage{amssymb}
\usepackage{epsfig}
%\usepackage{natbib}
\usepackage{color}
\usepackage{url}
\textwidth16.2cm
%\textheight24cm
%\topmargin-1.8cm
%\footskip1.3cm

\oddsidemargin-0.2cm
\evensidemargin0.2cm
\makeatletter
%%%%%%%%%%%%%%%%%%%%%%%%%%%%%% Textclass specific LaTeX commands.
% \copyrightyear{2007}
% \pubyear{2007}
% \newcommand{\lyxaddress}[1]{
%   \par {\raggedright #1 
%$   \vspace{1.4em}
 %  \noindent\par}
% }

%%%%%%%%%%%%%%%%%%%%%%%%%%%%%% User specified LaTeX commands.

%\usepackage{ae,aecompl}

\usepackage{babel}
\makeatother
\begin{document}
%\hspace{-0.5cm}
\thispagestyle{empty}
23 February 2021

\vspace{0.5in}

Dear Editor,

\vspace{0.5cm}

%\hspace{-0.5cm}

We would like to submit our manuscript, ``WOMBAT: A fully Bayesian global flux-inversion framework,'' for consideration by the \emph{Journal of the American Statistical Association -- Applications \& Case Studies}. The manuscript focuses on the important and difficult problem of estimating the sources and sinks of greenhouse gases (in our case CO$_2$) from remote-sensing instruments, and is the result of a long-standing collaboration between statisticians and atmospheric chemists. We develop the statistical machinery we believe is required to tackle a problem of this complexity, and also show that our estimates of carbon dioxide sources and sinks are often an improvement on those obtained by leading groups across the world.  As a result of the sophisticated inter-disciplinary nature of this work, we envisage it to have a lasting impact in both the applied statistics and the atmospheric chemistry communities worldwide.

The submission consists of the main manuscript (blinded and unblinded) and supplementary material (blinded by default). We also provide scripts that reproduce the results shown in the manuscript. However, for practical reasons, the raw data from the numerical atmospheric transport model have not been made available due to their sheer size (terabytes); instructions on how to generate these data are provided in  \texttt{README.md} in the supplementary code package. We have also stored the data offline, and the data are available to the reviewers on request. Intermediary data files, which are sufficient to reproduce the results shown in the manuscript, are a few gigabytes in size, and are available for download as detailed in \texttt{README.md} in the supplementary code package; a link to the data on our servers will be provided once there is no need to remain anonymous. More details on our data and code are given in the accompanying Author Contributions Checklist form.

%Since JASA adopts a double-blind peer-review process we have refrained from uploading our manuscript to \emph{arXiv}, but may do so in the future if and when it becomes critical that the work becomes public.

Please let me know if there is anything more you need from us to complete the submission.

 With my best wishes,

 \vspace{0.2in}

Andrew Zammit-Mangion

\vspace{-0.1in}

National Institute for Applied Statistics Research Australia (NIASRA)

\vspace{-0.1in}

School of Mathematics and Applied Statistics

\vspace{-0.1in}

University of Wollongong NSW 2522, Australia

\vspace{-0.1in}

e-mail: azm@uow.edu.au

\vspace{-0.1in}

tel.: +61 2 4221 5112


\end{document}
