\documentclass[english]{letter}
\usepackage[latin1]{inputenc}
\usepackage{amsmath}
%\usepackage{caption}
%\usepackage{subfig}
\usepackage{graphicx}
\usepackage{amssymb}
\usepackage{epsfig}
%\usepackage{natbib}
\usepackage{color}
\usepackage{url}
\textwidth16.2cm
%\textheight24cm
%\topmargin-1.8cm
%\footskip1.3cm

\oddsidemargin-0.2cm
\evensidemargin0.2cm
\makeatletter
%%%%%%%%%%%%%%%%%%%%%%%%%%%%%% Textclass specific LaTeX commands.
% \copyrightyear{2007}
% \pubyear{2007}
% \newcommand{\lyxaddress}[1]{
%   \par {\raggedright #1 
%$   \vspace{1.4em}
 %  \noindent\par}
% }

%%%%%%%%%%%%%%%%%%%%%%%%%%%%%% User specified LaTeX commands.

%\usepackage{ae,aecompl}

\usepackage{babel}
\makeatother
\begin{document}
%\hspace{-0.5cm}
\thispagestyle{empty}
04 October 2021

\vspace{0.5in}

Dear Editor,

\vspace{0.5cm}

%\hspace{-0.5cm}

We would like to submit our manuscript, ``Modelling, Fitting, and Prediction with Non-Gaussian Spatial and Spatio-Temporal Data using FRK'', for consideration by the \emph{Journal of Statistical Software} (JSS). The manuscript presents a major upgrade to \textbf{FRK}, an \texttt{R}  package for spatial/spatio-temporal modelling and prediction with large data sets, that allows it to model non-Gaussian data with an order of magnitude more basis functions than was previously possible. This upgrade is the result of popular demand and feedback from many of our users since the first upload to CRAN in 2017.  This new version of \textbf{FRK} addresses a large gap in the currently available software and, as our extensive case studies show, we envisage it to play an important role in spatial and spatio-temporal statistics in a variety of new application domains. 

An article on the first version of \textbf{FRK} was first submitted to JSS in June 2017, was conditionally accepted in 2018, and was published in 2021. Although the first article on \textbf{FRK} was published  recently, work on this upgrade has been ongoing for three years now, and this paper represents the culmination of the work. 

The submission consists of the main manuscript and the scripts that reproduce the results shown in the manuscript. Instructions on how to generate these results are provided in the file \texttt{README.md}. Where possible, we have strived to follow the guidelines provided by JSS, and have made every effort to ensure that the experience for reviewers is as smooth as possible. For example, we do not require high performance computing for reproducing the results, and we also provide a ``quick''  option for reviewers to be able to generate (inexact) figures and tables quickly.


Please let me know if there is anything more you need from us to complete the submission.




%\textbf{FRK} is an \texttt{R} package for spatial/spatio-temporal modelling and prediction with very large data sets. 
 %Key features of the package include its ability to handle both point-referenced and areal data simultaneously, and its solution to spatial change-of-support problems (whereby one wishes to make inference over supports that are different to the supports of observed data). 
% The main purpose of our new manuscript is to describe a major upgrade to \textbf{FRK} that allows it to model non-Gaussian spatial and spatio-temporal data. 
 %These data are ubiquitous in real-world applications, arising whenever counts are taken (e.g., Poisson, binomial, or negative-binomial data), the response is binary (e.g., Bernoulli data), or the domain of the response is restricted to the positive real line (e.g., gamma or inverse-Gaussian data).   
 %This new version of \textbf{FRK} addresses a large gap in the currently available software, and we envisage it to play an important role in spatial and spatio-temporal statistics moving forward. 




%Finally, we would like to acknowledge that this manuscript comes quite recently after the manuscript describing the first version of \textbf{FRK} was published. That manuscript was submitted in June 2017 and, although it was first conditionally accepted by JSS in September 2018, work on the second version began before then. Hence, this current manuscript is several years in the making. 


%Please let me know if there is anything more you need from us to complete the submission.

 With my best wishes,

 \vspace{0.2in}

Matthew Sainsbury-Dale

\vspace{-0.1in}

National Institute for Applied Statistics Research Australia (NIASRA)

\vspace{-0.1in}

School of Mathematics and Applied Statistics

\vspace{-0.1in}

University of Wollongong NSW 2522, Australia

\vspace{-0.1in}

e-mail: msdale@uow.edu.au



\end{document}
